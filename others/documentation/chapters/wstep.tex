Nowoczesne rozwiązania technologiczne pozwalają domowym użytkownikom na budowę systemów wbudowanych dostosowanych do ich potrzeb. Platformy takie jak \emph{Raspberry PI} czy \emph{Arduino}, w połączeniu z wieloma czujnikami dostępnymi na rynku, dają możliwości, które jeszcze kilka lat temu były niewyobrażalne. Projekt, którego dotyczy niniejsza dokumentacja, jest przykładem systemu, który został zrealizowany właśnie dzięki tego typu rozwiązaniom. Będzie on umożliwiał rozmowę ze sztuczną inteligencją udostępnianą przez \emph{Google} (\emph{Google Assistant}). Pozwala ona między innymi na sprawdzenie pogody, swojej skrzynki mailowej, kalendarza i innych usług udostępnianych przez \emph{Google}. Oprócz tego możemy zadać jej właściwe każde pytanie.


System zostanie rozbudowany o możliwość sprawdzenia stanu fizycznej skrzynki pocztowej. Ma ono następować poprzez stronę internetową lub komendę głosową, na którą urządzenie odpowie nam odpowiednim komunikatem. W tym celu zostanie stworzony serwer aplikacji, serwer webowy oraz osobny podsystem oparty na module \emph{ESP8266} pozwalający na zbadanie aktualnego stanu skrzynki. Komunikacja urządzeń będzie wymagała sieci WiFi oraz dostępu do Internetu.

Ten dokument składa się z sześciu rozdziałów. Pierwszy z nich stanowi niniejszy wstęp. W drugim zostaną omówione założenia projektowe. W trzecim można znaleźć informacje o wykorzystanych technologiach. W czwartym zostanie opisany projekt poszczególnych elementów systemu. W piątym znajdą się szczegółowe informacje dotyczące realizacji części projektowej. Ostatni rozdział zostanie poświęcony na uwagi i wnioski jakie wyniknęły podczas realizacji projektu.