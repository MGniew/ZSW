\section{Raspberry Pi - Google Assistant}

\subsection{Projekt fizyczny}

Na podstawie schematu (dostępnego w poprzednim rozdziale) został zbudowany prototyp urządzenia pełniącego rolę asystenta.

\begin{center}
\adjustbox{valign=t}{\includegraphics[width=10cm]{out.jpg}}
\end{center}
	
\begin{center}
\adjustbox{valign=t}{\includegraphics[width=10cm]{in.jpg}}
\end{center}

\subsection{Instalacja Google Assistant}

Podstawowa konfiguracja Google Assistant jest bardzo dobrze opisana na stronie \url{https://developers.google.com/assistant/sdk/overview} (zakładka "Python").

\subsection{Opis pliku wejściowego}

Zakładając projekt z wykorzystujący Google Assistant SDK dostajemy skrypt realizujący podstawową funkcjonalność asystenta:


\begin{lstlisting}[language=Python]
#!/usr/bin/env python

# Copyright (C) 2017 Google Inc.
#
# Licensed under the Apache License, Version 2.0 (the "License");
# you may not use this file except in compliance with the License.
# You may obtain a copy of the License at
#
#     http://www.apache.org/licenses/LICENSE-2.0
#
# Unless required by applicable law or agreed to in writing, software
# distributed under the License is distributed on an "AS IS" BASIS,
# WITHOUT WARRANTIES OR CONDITIONS OF ANY KIND, either express or implied.
# See the License for the specific language governing permissions and
# limitations under the License.


from __future__ import print_function

import argparse
import os.path
import json

import google.oauth2.credentials

from google.assistant.library import Assistant
from google.assistant.library.event import EventType
from google.assistant.library.file_helpers import existing_file


def process_event(event):
    """Pretty prints events.
    Prints all events that occur with two spaces between each new
    conversation and a single space between turns of a conversation.
    Args:
        event(event.Event): The current event to process.
    """
    if event.type == EventType.ON_CONVERSATION_TURN_STARTED:
        print()

    print(event)

    if (event.type == EventType.ON_CONVERSATION_TURN_FINISHED and
            event.args and not event.args['with_follow_on_turn']):
        print()


def main():
    parser = argparse.ArgumentParser(
        formatter_class=argparse.RawTextHelpFormatter)
    parser.add_argument('--credentials', type=existing_file,
                        metavar='OAUTH2_CREDENTIALS_FILE',
                        default=os.path.join(
                            os.path.expanduser('~/.config'),
                            'google-oauthlib-tool',
                            'credentials.json'
                        ),
                        help='Path to store and read OAuth2 credentials')
    args = parser.parse_args()
    with open(args.credentials, 'r') as f:
        credentials = google.oauth2.credentials.Credentials(token=None,
                                                            **json.load(f))

    with Assistant(credentials) as assistant:
        for event in assistant.start():
            process_event(event)


if __name__ == '__main__':
    main()
    
\end{lstlisting}

Powyższy kod tworzy obiekt asystenta (wykorzystując w tym procesie dane uwierzytelniające), po czym w pętli zaczyna przetwarzanie zdarzeń (przykładem zdarzenia jest początek konwersacji wywoływany słowami "Hey Google"). Zmieniając implementację metody process\_event możemy wpływać na zachowanie asystenta.


\begin{lstlisting}[language=Python]
def process_event(cp, event, assistant):
    """Pretty prints events.
    Prints all events that occur with two spaces between each new
    conversation and a single space between turns of a conversation.
    Args:
        event(event.Event): The current event to process.
    """
    if event.type == EventType.ON_CONVERSATION_TURN_STARTED:
        print()
        gpio.output(22, True)

    print(event)
    
    if event.type == EventType.ON_RECOGNIZING_SPEECH_FINISHED:
        try:
            if cp.read_command(event.args['text']):
                assistant.stop_conversation()
        except ValueError as e:
            print(e)
            
    if (event.type == EventType.ON_CONVERSATION_TURN_FINISHED and
            event.args and not event.args['with_follow_on_turn']):
        print()
        gpio.output(22, False)
\end{lstlisting}

Powyższy kod przedstawia zmodyfikowaną wersję metody process\_event. Najważniejszym elementem jest przechwycenie zdarzenia "ON\_RECOGNIZING\_SPEECH\_FINISHED", w którym można znaleźć nasze słowa zamienione na tekst (ang. Speech To Text). Dzięki temu, odpowiednio przetwarzając zdarzenie, możemy zaimplementować własne reakcje systemu. W tym celu stworzyliśmy klasę (opisaną dokładniej w dalszej części dokumentu) "CommandProcessor" (cp). Przyjmuje ona treść naszych słów i szuka odpowiedniej komendy do wywołania - w przypadku znalezienia takowej, "rozmowa" z asystentem jest przerywana (nie usłyszymy odpowiedzi od sztucznej inteligencji). W powyższej metodzie dopisaliśmy również reakcje na zdarzenia "ON\_CONVERSATION\_TURN\_STARTED" i "ON\_CONVERSATION\_TURN\_FINISHED" jest to odpowiednio zapalanie i gaszenie diody.

\subsection{Własne komendy}




