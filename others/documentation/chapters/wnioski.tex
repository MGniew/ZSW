W efekcie pracy nad projektem powstał działający prototyp systemu w pełni realizujący zadania przed nim stawiane. Dzięki możliwości definiowania własnych komend głosowych i dołączaniu kolejnych podsystemów, można rozbudować projekt do rozmiarów pozwalających na stworzenie np. inteligentnego domu. Jedną z niewielu wad, którą posiada system, jest brak funkcjonalności pozwalającej na zamianę pisma na mowę (ang. \emph{Text to Speech} - wszelkie słowne odpowiedzi urządzenia są uprzednio nagrane i odtwarzane w razie potrzeby. Rozwiązaniem problemu mogłoby być zastosowanie osobnego narzędzia implementującego tę funkcję, jednak istnieje prawdopodobieństwo, że \emph{Google Assistant} będzie posiadał taką możliwość w przyszłości. Kolejnym elementem jaki należy poprawić w projekcie jest stworzenie odpowiedniej obudowy dla głównego urządzenia - kartonowe pudełko jest mało estetyczne i podane na uszkodzenia.

Możliwości jakie dają współczesne technologie mogą być z powodzeniem wykorzystywane w wielu dziedzinach. Do stworzenia własnego systemu wbudowanego wystarcza podstawowa wiedza z zakresu programowania i elektroniki, co (wraz ze stosunkowo niskim kosztem podzespołów) pozwala na tworzenie domowym użytkownikom rozwiązań dostosowanych do ich potrzeb.  